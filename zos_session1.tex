\documentclass{report}

\title{z/OS Masterclass}
\author{Alexander Brown}

\begin{document}
\maketitle

% Chapter 1-2
\section{The New Mainframe}

Run as much as possible, as long as possible without disrupting service.

100\% CPU all of the time.

Run multiple, but isolated, operating systems concurrently.

Optimized for I/O.

96 Processors (spares, OS and Hardware specific), 3TB memory.

RAS -- Reliability, Avilabilty, Serviceability.

z/OS -- zero (downtime)/OS.

Software as reliable as customers expect on zOS

Shared Everywhere - Centralised Control.

Goal mode processing - transaction needs to finished within $x$ms - WLM (processer manager).

\subsection{Batch Job versus Real Time Transactions}

Batch -- Input data -> Application Program -> Output.

Real-time -- Query -> Application Program -> Reply.

SNA -- System Network Arcitecture (before TCP/IP).

VTAM -- Virtual Telecommunications Application ...

\subsection{Error codes}

IWASD - Information, Wearning, Application, ?, Disaster.

...DFH -- CICS

\subsection{Mainframe OSes}

\subsubsection{z/OS (MVS)}
Trimodal - switch between 24, 31 and 64bit.

\subsubsection{z/VSE Virtual Storage Extended}
Cut down version of z/OS

JCL different.

\subsubsection{z/TPF Transaction Processing Factory}
Around 50 customers use this. Used to be aircraft control system. Tuned to be ultra-fast.

\subsubsection{z/VM}
Hypervisor.

Two types:

z/BM -- z/OS, etc.

PRISM - Schedule processors and h/w

Two basic components:

Control Program (CP).

Conversational Monitor System (CMS), a single-user OS.

CP creates multiple VMs from real h/w resources.

Appears as if each VM has dedicated use of shared resources.

\subsubsection{zLinux}
Uses ASCII not EBCDIC.


\section{Hardware Software and LPARsi (Logical PARtitions)}

Devices addressed by device, but today is virtulized.

FIO control layer uses a control file IOCDS that translates physical IO addresses into devce numbers.

ESCON and FICON switch from CP to peripheral devices.

Sysplex are groups of LPARs.

CF (coupling Facility) is a high speed memory area which Sysplexes can share.

Data sets: Lists, Locks and Cache

Share CPUs over LPARs if wanted.

LPARs are an images of an OS, try and share as much as possible.

Controled mostly via HNC.

PR/SM - Hypervisor.

Up to 60(+?) LPARs.

can have a native LAPR on z/OS.

LPARs are inderpendant of each other but resources are shared.

Can run different versions.

\subsection{Processors}

General Central Processor (CP): Standards applications and workloads.

System Assist Processor (SAP): Schedule I/O Operations.

Intergrated Facility for Linx (IFL): Used exclusively by a Linux LPAR/Linux on VM.

zOS Application Assist Processor (zAAP): Provides for Java and XML workload offload.

zOS Intergrated Information Processor (zIIP): Used to optimize certain database worload functions and XML processing.

Intergrated Coupling Facility (ICF): Used exclusively bt the Coupling Facility Control Code (CFCC) providing resource and data sharing.

Spares: Used in the event of a processor failure.

\subsubsection{On Demand}
CBU -- Capacity Back Up

CUoD -- On/Off Capacity Upgrade on Demand

SubCapacity Licensing Charges.

\subsection{Clustering}

Has been done for several years. Basic shaing don't usoing DASD.

CTC (channel to channel)/GRS (Global Resource S..) rings. Exclusive Locks and Memory Locks handled by a GRS.

RACF -- User permissions and password controller.

\section{Parallel Sysplex}
Tightly oupled LPARs, co-operating.

Coupling Facility and CFCC allows data to be shared between LPARs quickly.

Up to 32 LPARs (8 bytes).

Backwards compatable, so long as no new features are used.

Clock shared. z/OS has an instruction to return the Clock. Guarenteed unique clock.

8 byte clock, now 12 bytes, but not all products use the clock format.

\subsection{Exploiters of the CF}

Authorized applications (CICS, MQ).

IMS DB

DB2

VSAM

\subsubsection{CICS}

CICS-Sysplex (CICSPlex).

CICS connectivety can be outside of a sysplex.



\end{document}
